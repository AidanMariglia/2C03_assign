\documentclass[12pt]{article}

\usepackage{fullpage}
\usepackage{booktabs}
\usepackage{graphicx}
\usepackage{hyperref}
\usepackage{enumitem}
\usepackage{listings}
\usepackage{amsmath}
\usepackage{amssymb}
\usepackage{amsthm}

\lstset{language=Java, basicstyle=\tiny, breaklines=true, showspaces=false,
showstringspaces=false, breakatwhitespace=true}

\title{2C03 Assignment 2}
\author{Aidan Mariglia}

\begin{document}
\maketitle


\section*{4.1.31}
Since we have V vertices and each edge will connected 2 vertices, the number of
possible edges is $\binom{V}{2}$. If the graph has E edges, than we will need to
select E edges from the group of unique edges we just described. The resulting
number of graph permutations is $\binom{\binom{V}{2}}{E}$. Using the binomial
co-efficient formula, we are left with $\frac{\frac{V!}{2!(V - 2)!}!}{E!(\frac{V!}{2!(V - 2)!} - E)!}$
possible graph permutations.

\section*{4.1.32}
Leveraging a hashtable, duplicate edges can be found in linear time.
Using a hash function that takes into account a tuple (v1, v2), where v1 and
v2 are the vertices connected by edge e, a simple boolean value can be stored
in the hash table. The edges of the graph can be scanned linearly, with each
edge being stored in the hashtable. If a collision occurs, then a parallel
edge has been found.

\section*{4.2.31}
To start, an boolean array of a size equal to the number of vertices in the 
graph is initialized. This will keep track of which vertices have been visited.
Another array is initialized to keep track of which vertices are part of the
strongly connected component. A depth first search is called on the vertex v.
If one of the adjacent nodes to the current node is a part of the strongly connected
component, the current node is added to the strongly connected component. To
find all of the strongly connected components in a digraph, call the modified
dfs that was previously described on each node within the graph once.

\section*{4.2.41}
To look for an odd cycle, first we must check if a graph is bipartite.
This can be done using BFS, colouring the source vertex red, and all of its
adjacent vertices blue. When arriving on a node, check if any of its adjacent
vertices are the same colour as it is, if not, colour its non-coloured adjacent
vertices blue if the current vertex is red, or red if the current vertex is blue.
A graph with an odd cycle will always have at least one vertex that has an adjacent
vertices which is the same colour.
(Source - https://www.geeksforgeeks.org/bipartite-graph/)

\end{document}
