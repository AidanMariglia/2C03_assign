\documentclass[12pt]{article}

\usepackage{fullpage}
\usepackage{booktabs}
\usepackage{graphicx}
\usepackage{hyperref}
\usepackage{enumitem}
\usepackage{listings}
\usepackage{amsmath}
\usepackage{amssymb}
\usepackage{amsthm}

\lstset{language=Java, basicstyle=\tiny, breaklines=true, showspaces=false,
showstringspaces=false, breakatwhitespace=true}

\title{2C03 Assignment 2}
\author{Aidan Mariglia}

\begin{document}
\maketitle

In collaboration with:
\begin{itemize}[noitemsep]
    \item Ben Dubois
    \item David Moroniti
    \item Declany Young
    \item Nathan Isaac Uy
\end{itemize}
%%3.2.15 3.2.24 3.2.36 3.3.38
%$ 3.4.15 3.5.25
\section*{3.2.15}

\begin{enumerate}[label=(\alph*)]
\item E $\rightarrow$ Q
\item E $\rightarrow$ Q
\item E $\rightarrow$ Q
\item E $\rightarrow$ Q $\rightarrow$ J
\item E $\rightarrow$ D $\rightarrow$ Q $\rightarrow$ J $\rightarrow$ M $\rightarrow$ T $\rightarrow$ S
\item E $\rightarrow$ D $\rightarrow$ A $\rightarrow$ Q $\rightarrow$ J $\rightarrow$ M $\rightarrow$ T $\rightarrow$ S
\end{enumerate}

\section*{3.2.24}

Building a BST entails using the 'put' method on each member of
a list of input values, in our case this list is N members long.
In the best case, we construct a balanced BST, and our function calls
for 'put' cost log(N) compares, as one compare is required at each level in
the tree, and a ballanced tree will have log(N) levels. From this it is easy to see 
that for building a BST the run time has a lower lower bound of Nlog(N), 
as 'put' will always be called N times in order to insert N values 
into the BST, and as previously mentioned, the best case runtime for put is 
proportional to log(N).

\section*{3.2.36}

It is possible to create a nonrecursive version of keys using space proportional
to the trees height. This is done by considering vertical slices of the tree,
containing all nodes within the range. A stack is used to temporarily store
values from the tree. First the root of the tree is pushed onto the stack,
along with all of its left children (Stopping at smallest left child
withing the specified range). Then, the top node is popped, its key is added
to the queue, and if it has a right child, its right child and all of its 
left children are pushed onto the stack. If the node does not have a right
child, the next value on the stack is popped. This process stops when the max
value in the range is reached, and the queue is returned.

\section*{3.3.38}

For every sequence of N values there exists a finite number of BST permutations
that preserve the properties of BST's, (left child is less than right child).
There are two extreme permutations which are essentially linked lists (maccaroni),
a list in descending order with the greatest value at the root and every right
child equal to null, and a list in ascending order with the smallest value at
the root and every left child being null. A 'descending list' tree can be 
made into an 'ascending list' tree using repeated right rotations. In the 
process of changing from one extreme to the other, the tree exists as all
of the permutations in between, meaning it is possible to move from one
permutation to the other using only rotations.

\section*{3.4.15}

It would take  $\sim \frac{N^2}{2}$ compares in the worst case.

\section*{3.5.25}

To prevent such conflicts, the University could store the lecture times
in a hash table, with the lecture time itself being the value that is hashed.
In the case of the same value being inserted twice, the same hash will be generated
and a collision will occur. In the case of a "collision" it will be clear that
it is not possible to schedule a class at this time.

\end{document}