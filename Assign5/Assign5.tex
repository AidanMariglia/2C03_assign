\documentclass[12pt]{article}

\usepackage{fullpage}
\usepackage{booktabs}
\usepackage{graphicx}
\usepackage{hyperref}
\usepackage{enumitem}
\usepackage{listings}
\usepackage{amsmath}
\usepackage{amssymb}
\usepackage{amsthm}

\lstset{language=Java, basicstyle=\tiny, breaklines=true, showspaces=false,
showstringspaces=false, breakatwhitespace=true}

\title{2C03 Assignment 5}
\author{Aidan Mariglia}
%%5.2.8, 5.3.11 5.3.21
\begin{document}
\maketitle

\subsection*{5.2.8}

floor():

\noindent Recursively traverse all branches of the trie, using a counter to 
keep track of the number of nodes passed at each step. When the value stored
at a node is not null (i.e. it indicates the end of a key). Add that key to a
list. When the counter exceeds the length of the target key, return. Sort the
returned list, and return from the method the longest key which is not the
target key.

\medskip
\noindent ceiling():

\noindent Recursively traverse all branches of the trie, using a counter to 
keep track of the number of nodes passed at each step. When the counter exceeds
or is equal to the length of the target node, and the value stored at a node is
not null (i.e. indicating the end of a key), add this ket to a list and return.
Sort the returned list, and return the smallest value that is not the target key.

\medskip
\noindent rank():

\noindent Recursively traverse all branches of the trie, keeping track of the
number of nodes passed. Each time the value of the counter is less than the
length of the target key, and the value at the node is not null, add that key
to a list. When the counter exceeds or is equal to the length of the target
key, return. Then, simply count the number of items in the list.

\medskip
\noindent select():

\noindent Recursively traverse all branches of the trie. When you arrive
at a key, use rank() to determine its rank. If it matches the target rank,
return that key.

\medskip
\noindent

\subsection*{5.3.11}

Let P represent the pattern string, and T represent the text. First,
any worst case for Boyer-Moore will require the alphabets used in P and T to be the same, 
this makes the mismatched character heuristic useless, as there will never be a character
in T that is not also in P. Secondly, T will have many instances of P, however
there will also be many instances that are slightly different that P. For example,\\

\noindent T = aabababaabababaabababaabababaababab\\
P = babab\\

\subsection*{5.3.21}
To modify the Rabin-Karp algorithm to allow for a middle character "wildcard"
is to modify the hash function used to exclude the middle character. Then,
when the pattern is aligned with the substring in the text, check all characters
excluding the middle.



\end{document}