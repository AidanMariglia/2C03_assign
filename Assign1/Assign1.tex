\documentclass[12pt]{article}

\usepackage{fullpage}
\usepackage{booktabs}
\usepackage{graphicx}
\usepackage{hyperref}
\usepackage{enumitem}
\usepackage{listings}

\lstset{language=Java, basicstyle=\tiny, breaklines=true, showspaces=false,
showstringspaces=false, breakatwhitespace=true}

\title{2C03 Assignment 1}
\author{Aidan Mariglia}

\begin{document}
\maketitle

In collaboration with:
\begin{itemize}[noitemsep]
  \item Ben Dubois
  \item David Moroniti
  \item Declan Young
  \item Nathan Isaac Uy
\end{itemize}


\section*{1.3.45}
  

To check if stack underflow ever occurs it is important to ensure that
the number of consecutive pop operations never excedes the current height
of the stack. To achieve this an integer counter can be used. The sequence
can be checked, with every integer found incrementing the counter and every
'-' found decrementing the counter. If the counter is ever a negative value,
then stack underflow has occured. The memory used by this algorithm is
independent of N as only one variable is instantiated to track the current
height of the stack. (the char array n is only used to simulate an input stream).


The second part of this question requires a more complex approach. All of the
values from $1$ to $N-1$ are iterated through, and the value of the output
permutation is checked at each index. In cases where the stack is empty, or 
the value of the permutation at that index is greater than the index, values
starting at the index and ending at the value stored at that index are pushed
onto the stack. If the value on the top of the stack is equal to the current 
permutation value, then the stack is popped. If neither of these conditions are
met, then it is not possible for the output permutation to exist. This is because
values are pushed onto the stack in order from least to greatest, meaning if the
value on top of the stack is greater than the current permutation value at that
index, then the desired value is somewhere else lower on the stack, and since we 
are only using one stack and the values on top of the desired values cannot be 
temporarily stored elsewhere, the specific permutation cannot exist.
This is completed in linear time as the worst case would involve the permutation
having the value $N-1$ as the first value, causing N pushes to the stack, however
when the next value of the permutation was checked it would have to be lower than
the value on top of the stack, meaning that the loop would terminate and the method
would return false.


\newpage
\noindent \lstinputlisting[]{./StackUnderflow.java}



\section*{1.4.5}

\begin{enumerate}[label=(\alph*)]

  \item $n + 1 = \sim  n$
  \item $1 + \frac{1}{n} = \sim 1$ (As $n$ approaches $\infty$ $\frac{1}{n}$
  approaches $0$)
  \item $(1 + \frac{1}{n})(1 + \frac{2}{n}) = \sim 1$ (Same as above for
  $\frac{1}{n}$ and $\frac{2}{n}$)
  \item $2n^3 - 15n^2 + n  = \sim 2n^3$ ($2n^3$ most significant term)
  \item $\frac{\log 2n}{\log n} = \sim 1$ ($\log{2n} = \log 2 + \log n$,
  we can ignore $\log 2$ $\frac{\log n}{\log n} = 1$)
  \item $\frac{\log (n^2 + 1)}{\log n} = \sim 2$ (Ignore + 1, $\log n^2 = 2\log n$,
  $\frac{2 \log n}{\log n} = 2$)
  \item $\frac{n^{100}}{2^n} = \sim 0$ ($2^n$ "blows up" faster than $n^{100}$)

\end{enumerate}

\section*{1.4.6}

\begin{enumerate}[label = (\alph*)]
  \item The iterator sum++ will be the operation used to track run time
  as it occurs most. The outer for loop starts with n = N and halves N every
  iteration. The inner for loop runs n times. Meaning the sum variable will be
  iterated $N + \frac{N}{2} + \frac{N}{4} + \frac{N}{8} ...$ times. This is a
  geometric series, the sum of which is $2N-1$. Therefore the order of growth
  is $N$. 
  \item This loop structure is largely the same as the previous question, except
  rather than halving the value of every time, it is doubled. However the inner
  variable is iterated the same number of times, meaning the order of growth is $N$.
  \item The outer loop will run $\log N$ times. The inner loop always runs
  $N$ times. Therefore the order of growth $N\log{N}$.

\end{enumerate}

\section*{1.4.12}

The algorithm chooses one of the arrays to iterate through. It first checks the
simpliest case, the equality of the two arrays at the same index. If they are 
equal, than it prints the duplicate. If not it checks if the value in the 
array we are iterating through is greater than that off the other array. If it
greater then the algorithm checks successive values in the second array until 
either it is out of the bounds of the array or the value in the second array is
greater then the first. The third case is the same as the second except it goes
in the opposite direction, checking values below. The algorithms running time is
proportional to N because the length of the arrays are the same and the arrays 
are both sorted. This means there are a limited number of times the inner while
loops can run before they will terminate. For example, in the worst case the first
index of the first array would have a duplicate that is the final index of the
second array. Since every number in the second array must then be smaller than
the first number in the first array, the while loops will never run, and the 
total running time will be $N + a$ where a is some constant.



\noindent \lstinputlisting[]{./SortedDuplicates.java}



\section*{1.4.17}

To find the furthest pair in the unsorted array we need to find the largest and
smallest number. Iterating through the array once and checking for the minimum
and maximum number ensures the furthest pair is returned in linear time.

\noindent \lstinputlisting[]{./FurthestPair.java}

\section*{1.4.34}
To solve the Hot-Cold problem start with 2 initial guesses, at 1 and N. Calculate
the distance between the first guess, 1 and the desired number, and the second
guess N, and the desired number. If the second guess is closer we say the guess
is "hot" or closer than the previous guess. If not it is cold. This allows us 
to split our range of numbers in half. Depending on if the guess was hot or cold,
we can guess again either at the "top" or "bottom" of our new range, continously
spliting our range in half until the guess is equal to the desired number.
Since with each iteration we halve the number of elements that could possibly
contain the number, the run time is $\log N$.
\section*{1.5.14}

This algorithm is very similar to the weighted quick-union. The only main
difference is how the height is incremented compared to how the size is 
accumulated in weighted quick-union. With this implementation the shorter 
tree is always merged to the taller tree, ensuring that the height of a tree
only increases when it is merged with a tree of the same height. The upperbound
for the height of a tree is $ \log N$ because the worst case for tree height has
all of the member in one set, and each parent in the tree has two children. This
is the case as to increase the height of the tree, trees of matching heights 
must be merged. This means every time the height increases, the number of members
doubles. Or in other words we can only have $\log N$ number of doublings, so the
max height is $\log N$.


\noindent \lstinputlisting[]{./WeightedUF.java}
\section*{1.5.15}

The worst case occurs for a weighted quick union when the maximum number of
height increases occur. Since the height only increases when trees of the
same size are merged. Following a simple code trace we see that this causes 
the levels closer to the middle to have more nodes stored at them, increasing 
in the same way binomial co-efficients do, as you grow the worst case tree you
see the pattern 1, 1 1, 1 2 1, 1 3 3 1. The same as different levels of binary
co-efficients.

Since we can see that the number of nodes at certain levels of the tree resemble
binomial co-efficients and we also know that binomial co-effficients are 
mirrored across the center, we can infer that the average depth of a node in
a worst case tree will also be in the middle, or rather have a depth of $\frac{N}{2}$



\end{document}